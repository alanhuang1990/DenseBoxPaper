\section{Introduction}

As one of the core problems in image understanding, object detection has some significant breakthrough recent years after the widely applied deep convolutional neural networks. Before the success of convolutional neural networks, the best performing detectors in many benchmarks were based on a combination of handcrafted image feature such as HOG~\cite{dalal2005histograms} , SIFT~\cite{lowe2004distinctive} , and the Fisher vector~\cite{cinbis2013segmentation} , etc. These systems such as deformable part-based models (DPM)~\cite{felzenszwalb2010object} use sliding window framework to apply classifier at every object locations and scales. Recently, the convolutional neural network based methods such as R-CNN~\cite{girshick2014rich,girshick2015fast} bring a revolution on the field of object detection , providing a remarkable gain in detection accuracy compared to classic sliding window approaches. Conceptually, R-CNN contains two phases. First, region proposal methods are used to generate potential bounding boxes in the image. Then, a convolutional classifier is applied to each proposed bounding box. 

However, the different stages in R-CNN pipeline cannot be optimized jointly, and to classify thousands of proposal bounding boxes of one image, it usually requires half a minute. The second issue has been solved in the latest incarnation of R-CNN, the Faster R-CNN~\cite{ren2015faster} which shares the convolutional feature computation among different regions. Then the proposal generation becomes the new bottleneck. Recently some methods try to solve the first issue by one of the following two ways: 1) The Faster-R-CNN trains a region proposal network shared with CNN classifier; 2) YOLO~\cite{YOLO}  presents a single network end-to-end optimized directly on detection performance, and Lenc et al.~\cite{LencV15} proposed a simplified SPP-CNN~\cite{} that does not need proposal generation. All of them focus on acceleration of testing, and developing an end-to-end framework for detection. 

In this work, we focus on one question: To what extent can an one-stage CNN-based detection system perform? Although similar to YOLO~\cite{YOLO}, MultiBox~\cite{} and OverFeat~\cite{sermanet2013overfeat}, our system is more carefully designed for a set of specific problems such as face detection and car detection. Unlike general object detection in PASCAL VOC or ImageNet, the target objects like faces and cars could be very small but crucial to real world application ({\em i.e.} self-driving car). However, general proposal based detection methods could fail due to the the small resolution of objects. Our system is designed end-to-end to detect objects over all possible locations and scales in an image. 

To this end, we present a novel fully convolutional neural network based object detector, called DenseBox, that does not require proposal generation and is able to be optimized end-to-end from images.  Owning to the FCN structure,  the performance of DenseBox cen be easily improved by multi-task learning with landmark localization of landmark information is available.  In the evaluation we find that with the help of landmark localization and model ensemble, our method results in the best performance on MALF(Multi-Attribute Labelled Faces)~\cite{faceevaluation15} detection dataset. The results on KITTI~\cite{Geiger2012CVPR} car detection task are still competitive performance compared to method using stereo information. All our experiments shows that the purely fully convolutional networks for object detection can work very well with careful designed model. 

\section{Related Work}

 

% This is the new version, which reviewed both nn based detection in the 1990s and the current RCNN.

The application of neural networks for detection tasks such as face detection has a long history. The first work may date back to early in 1994 when Vaillant et al.\cite{vaillant1994original} proposed to train a convolutional neural network to detect face in image window.  Later in 1996 and 1998 Rowley et al.\cite{rowley1998neural,rowley1998rotation} presented many neural network based face detection system to detect upright frontal face in image pyramid. There is no way to compare the performance of those ancient detectors with today’s detection systems on face detection benchmarks. Even so, they are still worth revisiting, as we find many similarities in design with our DenseBox. 

Currently, most state-of-the-art object detection approaches\cite{ouyang2014deepid, li2015convolutional, erhan2014scalable,girshick2015fast}rely on R-CNN, which divides detection into two steps: salient object proposal generation and region proposal classification. Several recent works such as YOLO and Faster R-CNN have jointed region proposal generation with classifier in one stage or two stages. It is pointed out by \cite{farfade2015multi} that R-CNN with general proposal methods designed for general object detection could results in inferior performance in detection task such as face detection, due to loss recall for small-sized faces and faces in complex appearance variations. They share similarities with our method, and we will discuss them with our method in more detail in later context.  
 
